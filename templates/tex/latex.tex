\documentclass[11pt]{article}
\usepackage{fullpage}
\usepackage{parskip}

% algorithms
\usepackage{algorithm}
\usepackage[noend]{algpseudocode}

% maths essentials
\usepackage{amsmath}
\usepackage{amssymb}
\usepackage{bm} % bold math

% optionals
\usepackage{xcolor} % color stuff e.g. colorbox

% matrices and calculus
\newcommand{\br}[1]{\bm{\mathrm{#1}}} % shortcut for matrices
\newcommand{\numberthis}{\addtocounter{equation}{1}\tag{\theequation}}
\newcommand{\reason}[1]{\tag*{\{ #1 \}}} % shortcut for tagging reasoning
\newcommand{\trans}{^\mathrm{T}} % shortcut for transpose
\newcommand{\norm}[1]{\lVert#1\rVert} % ||v||
\renewcommand{\d}{\mathrm{d}} % d for derivatives / integral
\newcommand{\dydx}[2]{\frac{\d #1}{\d #2}} % derivative
\newcommand{\dydxn}[3]{\frac{\d^{#1} #2}{\d #3^{#1}}} % higher order derivative
\newcommand{\pydx}[2]{\frac{\partial #1}{\partial #2}} % partial derivative

% todo
\newcommand{\todo}{\colorbox{red!80}{\textbf{Was unable to solve}}}

% allow align*s to be broken over lines
\allowdisplaybreaks

\title{\LaTeX template}
\author{Chua Hou}
\date{}
\begin{document}

\maketitle

\section*{Question 5.1}

\subsection*{Part (a)}

Using the partial chain rule, we note that (setting
$\bm{g}(\bm{y}) = \bm{\mathrm{M}y}$):
\begin{align*}
	\dydx{f(\br{M}\bm{y})}{\bm{y}} (\bm{y_n}) &=
	\dydx{(f \circ \bm{g})}{\bm{y}} (\bm{y_n}) \\
	&= \br{J}(\bm{g})\trans
		\left(\dydx{f}{\bm{x}} \circ \bm{g}\right) (\bm{y_n})
		\reason{partial chain rule} \\
	&= \br{MJ}(\bm{y})\trans
		\dydx{f}{\bm{x}}(\bm{\mathrm{M}y}) \\
	&= \br{MI} \dydx{f}{\bm{x}}(\bm{x_n}) \\
	&= \br{M} \dydx{f}{\bm{x}}(\bm{x_n}). \numberthis \label{eqn1}
\end{align*}
Hence $\bm{x_n}$ can be computed directly using the preconditioned iteration.

Gradient descent for $\operatorname{arg\,min}_{\bm{y}} f(\bm{\mathrm{M}y})$
yields
\begin{align*}
	&&\bm{y_{n+1}} &= \bm{y_n} -
		\alpha_n \dydx{f(\br{M}\bm{y})}{\bm{x}} (\bm{y_n}) \\
	&\Rightarrow&\br{M}^{-1} \bm{x_{n+1}} &= \br{M}^{-1} \bm{x_n} -
		\alpha_n \br{M} \dydx{f}{\bm{x}} (\bm{x_n})
		\reason{substituting $\bm{x_n} = \br{M} \bm{y_n}$
			and \eqref{eqn1}} \\
	&\Rightarrow&\bm{x_{n+1}} &= \bm{x_n} -
		\alpha_n \br{M}^2 \dydx{f}{\bm{x}} (\bm{x_n})
		\reason{multiplying through by $\br{M}$}
\end{align*}
which proves ($*$).

\section*{Question 2}

We modify BFS to check if there are edges between vertices of depths of the
same parity. Since $G$ is a directed graph that is strongly connected, we are
certain that $G$ contains a cycle. Hence we only need to check that at least one
edge between vertices of the same depth parity exist to know that an odd cycle
exists.

\begin{algorithm}
\caption{Question 2}
\begin{algorithmic}[1]
	\Procedure{OddCycle}{$V, E$}
	\State $s \gets V[0]; d[s] \gets 0$
	\Comment{Arbitrary source}
	\For{$u \in V \setminus{\{s\}}$}
		$d[u] \gets \infty$
	\EndFor
	\State $Q \gets \varnothing$; \Call{Enqueue}{$Q, s$}
	\Comment{Enqueue only source}
	\While{$Q$ is not empty}
		\State $u \gets \Call{Dequeue}{Q}$
		\For{$v \in \textit{Adj}[u]$}
			\If{$d[v] = \infty$}
				\Comment{New node, set distance and enqueue}
				\State $d[v] \gets d[u] + 1$
				\State \Call{Enqueue}{$Q, v$}
			\ElsIf{$d[u] \equiv d[v] \pmod{2}$}
				\Comment{Visited node, check parity of depths}
				\State \Return \textsc{True}
			\EndIf
		\EndFor
	\EndWhile
	\State \Return \textsc{False}
	\Comment{Odd cycle not found}
	\EndProcedure
\end{algorithmic}
\end{algorithm}

\end{document}
